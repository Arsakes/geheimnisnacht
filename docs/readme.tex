\documentclass[10pt, english]{article}

\usepackage[T1]{fontenc}
\usepackage[utf8]{inputenc}
\usepackage{graphicx}
\usepackage{tikz}
\usetikzlibrary{arrows,automata}
%------------------------DEFINICJA STRUKTURY DO TKIZA------------------------------
\tikzset{
    pretile/.style={
        rectangle, minimum size=16mm, draw=black!100
    },
    tile/.style={
        pretile, 
        append after command={
            [every edge/.append style={
                black!100,
                shorten >=\pgflinewidth,
                shorten <=\pgflinewidth,
            }]
           (\tikzlastnode.north west) edge (\tikzlastnode.south east)
           (\tikzlastnode.north east) edge (\tikzlastnode.south west)
        }
    }
}
%---------------------------------------------------------
\author{Piotr<elminister@interia.pl>}


\begin{document}
\section{Developer Guide}
\subsection{Localisation Rules}
For the event localisation we use two kinds of conventions one for
completly new events, and for altered vanilla events (if there is need for them
to changed the localisation).
\paragraph{New warhammer events},
for that kind of events we always use namespaces, therefore 
naming scheme is as below: \\
\textit{EVTNAME\_<NAMESPACE>\_<event\_id>}
\\
\textit{EVTDESC\_<NAMESPACE>\_<event\_id>}
\\
\textit{EVTOPT\_<NAMESPACE>\_<eventid><A,B,C,D...}\\
We write namespace with \textbf{CAPITAL LETTERS}

\paragraph{Vanilla events}, for that we use:\\
\textit{EVTNAME\_WH\_<event\_id>}
\\
\textit{EVTDESC\_WH\_<event\_did>}
\\
\textit{EVTOPT\_WH\_<event\_id><A,B,C,D...}\\
The string within brackets \textit{<>} has to be filled with proper thing.\\
\textbf{REMEMBER}
There is only one file where new event localisation should be stored:
\textit{localisation\ events.csv}, please keep the localisations for
one namespace in block, followed by mark \textit{\# namespace name}
\\
\paragraph{Other localisation rules}
Don't create new files if not necsesary, search for the appropriate localisation file
and add the string in following block.
Commonly used files:
\begin{itemize}
\item modifiers - common.csv, section \# modifiers
\item traits - wh\_traits.csv
\item decisions - decisons.csv
\item various interface strings - common.csv or common\_religious\_cultural.csv
\end{itemize}

\newpage
\section{Core mechanics changes}
\section{Additional mechanics}
\subsection{CBs}
Test
\paragraph{chaos\_incrusion cb} - defined to represent large coordinated attack 
of chaotic forces.
Connected with event chain that applies bonuses to armies of the person that leads a 
chaos incrusion.
Can be used in sequence and for it to work \textbf{doom\_counter} must be high.
Characteristics:
\begin{itemize}
\item use decreases chances for other rulers to use other cbs and increases chances to use this one
\item available for high doom counter
\item all order religions can join the defender
\item all chaotic religion character can join the attacker
\item success/failure increasess/decreases the doom counter
\item apply special modifier to province that decreases 
\item ability to target empires and kingdoms both
\item imprisoning the enemy wouldn't lead to end of war? faction that rebels?
\item make it impossible to raise levies or anything of that sort
\item ensure that after taking the tile most of the old vassals would be converted or killed
\end{itemize}

\paragraph{possible implementation}
\begin{itemize}
\item base: holy war scheme,crusade scheme,regular cb
\item modifier to hinder ability for recruitment from provinces, along with set of events
      to change it
\item faction adjustment that wants independence in case of lost war
\item or a way to kill 
\item on\_defender\_leader\_death - choose new heir to respectful title the title
\end{itemize}
\end{document}
